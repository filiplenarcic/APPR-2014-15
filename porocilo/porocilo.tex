\documentclass[11pt,a4paper]{article}

\usepackage[slovene]{babel}
\usepackage[utf8x]{inputenc}
\usepackage{graphicx}
\usepackage{pdfpages}
\usepackage{hyperref}

\pagestyle{plain}

\begin{document}
\title{Poročilo pri predmetu \\
Analiza podatkov s programom R}
\author{Filip Lenarčič}
\maketitle

\section{Izbira teme}

V mojem projektu bom analiziral statistiko klubov severno-ameriške hokejske lige NHL.  V analizi bom primerjal statistične spremenljivke ekip ter na podlagi tega ugotavljal katere pozitivno in katere negativno vplivajo na uspešnost. Analiziral bom število strelov, ki jih je imela določena ekipa v odvisnosti od števila zadetkov in tako ugotovil katera ekipa je najučinkovitejša v napadu ter tudi v obrambi in na podlagi tega uvrstil ekipe glede na kakovost napada in obrambe. Prav tako bom analiziral igralce ekipe Chicago Blackhawks ter jih razvrstil v kategorije na podlagi tega ali je strelec vrhunski, dober, povprečen ali slab (urejenostna spremenljivka).  Učinkovitost igralcev ekipe Chicaga bom ugotovil na podlagi golov, asistenc in točk, ki jih je določen igralec dosegel v odvisnosti od števila odigranih tekem.
Cilj projekta je s programom RStudio analizirati podatke klubov in igralcev severno-ameriške hokejske lige NHL ter spoznati osnovne veščine programiranja v R-u. Podatke za moj projekt bom dobil na spletni strani NHL-ja (\url{http://www.nhl.com/ice/teamstats.htm}).


\section{Obdelava, uvoz in čiščenje podatkov}

V drugi fazi projekta sem uvozil dve tabeli v csv obliki. Ena izmed njiju zajema statistične podatke o igralcih, druga pa o ekipah. V tabeli igralcev sem tudi dodal nov stolpec z urejenostno spremenljivko in igralce razporedil v kategorije Vrhunski strelec, Dober strelec, Povprečen strelec in Slab strelelc. Na podlagi teh podatkov sem izdelal štiri grafe. Prvi graf prikazuje število zmag, ki so jih dosegli vsi klubi. Iz grafa je razvidno, katere ekipe so imele uspešno sezono in katere ne. Pri drugem grafu sem primerjal odstotek izkoriščenih Power-play priložnosti med desetimi ekipami lige. Tretji graf prikazuje povprečno število golov na tekmo, pri tem pa so bile v graf vklučene vse ekipe. Pri četrtem grafu pa sem se osredotočil na ekipo iz Chicaga ter analiziral število doseženih golov vsakega posameznika. Podatke sem pridobil iz spletne strani \url{http://www.nhl.com/ice/teamstats.htm}. 

Vsi moji grafi so zbrani tukaj:
\includepdf[pages={1-4}]{../slike/grafi.pdf}


\section{Analiza in vizualizacija podatkov}

V tretji fazi sem analiziral že doslej zbrane podatke in le te prikazal na nekaterih zemljevidih. Pri tem sem na novo uvozil tabelo NHLcities, v kateri najdemo podatke o zemljepisnih dolžinah in širinah severno-ameriškim mest, ki gostijo tekme lige NHL. Podatke o koordinatah sem pridobil na spletni strani \url{http://en.wikipedia.org/wiki/National_Hockey_League}. Na zemljevid ZDA, na katerem sem zaradi preglednosti odstranil Aljasko, Deviške otoke, Portoriko in Havaje, sem tako označil mesta, ki gostijo ligo NHL.
Za prikaz divizij v katere so uvrščeni klubi, sem se odločil tabeli uscapitals.csv dodati nov stolpec (nova tabela: capitals.csv), ki pove iz katere divizije je klub. Divizije so štiri in sicer: Pacific, Central, Metropolitan in Atlantic. Po izrisu zemljevida so tako zvezne države v katerih so doma klubi ločeni barvno po divizijah. 

Naslednja dva zemljevida ZDA prikazujeta število vseh zbranih točk v sezoni 2013/14 in število zadetih golov na tekmo v sezoni 2013/14 razvrščeno po zveznih državah. Za prikaz teh podatkov sem ponovno uvozil tabelo (goal.csv). Podatke iz tabele sem kopiral v zemljevid usastates. Na obeh zemljevidih opazimo da prevladujeta kluba Chicago in Boston, predtsvnika zveznih držav Illinois in Massachusetts. Po učinkovitosti sledi California, ki je domovina treh NHL klubov, ki so po oceni statističnih podatkov imeli kar zavisljivo sezono, kar se tiče uspešnosti. Med njimi je smiselno izpostaviti zmagovalce Stanleyjevega pokal klub Los Angeles Kings. Na drugi strani pa z slabšimi rezultati izstopata Florida in Tennessee.

Na zemljevisu sveta sem želel prikazati številčnost igralcev lige po narodnosti. Intenzivnejše barve pomenijo večje število NHL igralcev v določeni državi. Uvozil sem tabelo nacionalnost.csv, podatke pa pridobil iz sledečega spletnega vira: \url{http://www.quanthockey.com/nhl/nationality-totals/nhl-players-2013-14-stats.html}. Na zemljevidu so tako obarvane države s vsaj enim predstavnikom, ki igra v ligi NHL. Med državami prednjačijo najštevilčnejši Kanada in ZDA, sledijo Švedska, Finska ter nekatere slovanske države kot so Rusija Češka in Slovaška. Z Anžetom Kopitarjem, aktualnim prvakom lige, pa ima svojega edinega predstavnika tudi Slovenija. Iz zemljevida lahko tudi opazimo, da je hokej in s tem posledično najmočnejša hokejska liga NHL povezana z Evropo in Severno Ameriko, v slednji tudi najpriljubnejša kot ta vrsta športa.


\includegraphics{../slike/divizije.pdf}

\includepdf[pages={1-2}]{../slike/state.pdf}

\includegraphics{../slike/svet.pdf}

\end{document}
