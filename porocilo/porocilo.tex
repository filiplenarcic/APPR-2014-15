\documentclass[11pt,a4paper]{article}

\usepackage[slovene]{babel}
\usepackage[utf8x]{inputenc}
\usepackage{graphicx}

\pagestyle{plain}

\begin{document}
\title{Poročilo pri predmetu \\
Analiza podatkov s programom R}
\author{Filip Lenarčič}
\maketitle

\section{Izbira teme}

V mojem projektu bom analiziral statistiko klubov severno-ameriške hokejske lige NHL.  V analizi bom primerjal statistične spremenljivke ekip ter na podlagi tega ugotavljal katere pozitivno in katere negativno vplivajo na uspešnost. Analiziral bom število strelov, ki jih je imela določena ekipa v odvisnosti od števila zadetkov in tako ugotovil katera ekipa je najučinkovitejša v napadu ter tudi v obrambi in na podlagi tega uvrstil ekipe glede na kakovost napada in obrambe. Prav tako bom analiziral igralce ekipe Chicago Blackhawks ter jih razvrstil v kategorije na podlagi tega ali je strelec vrhunski, dober, povprečen ali slab (urejenostna spremenljivka).  Učinkovitost igralcev ekipe Chicaga bom ugotovil na podlagi golov, asistenc in točk, ki jih je določen igralec dosegel v odvisnosti od števila odigranih tekem.
Cilj projekta je s programom RStudio analizirati podatke klubov in igralcev severno-ameriške hokejske lige NHL ter spoznati osnovne veščine programiranja v R-u. Podatke za moj projekt bom dobil na spletni strani NHL-ja (http://www.nhl.com/ice/teamstats.htm).


\section{Obdelava, uvoz in čiščenje podatkov}

V drugi fazi projekta sem uvozil dve tabeli v csv obliki. Ena izmed njiju zajema statistične podatke o igralcih, druga pa o ekipah. V tabeli igralcev sem tudi dodal nov stolpec z urejenostno spremenljivko in igralce razporedil v kategorije Vrhunski strelec, Dober strelec, Povprečen strelec in Slab strelelc. Na podlagi teh podatkov sem izdelal tri grafe. Prvi graf prikazuje število zmag, ki so jih dosegli vsi klubi. Iz grafa je razvidno, katere ekipe so imele uspešno sezono in katere ne. Pri drugem grafu sem primerjal povprečno število golov, ki so jih dosegle ekipe. Primerjal sem deset ekip lige. Pri tretjem grafu pa sem se osredotočil na ekipo iz Chicaga ter analiziral število doseženih golov vsakega posameznika. Podatke sem pridobi liz spletne strani http://www.nhl.com/ice/teamstats.htm. 

\includepdf[pages=(1, 3)]{../slike/grafi.pdf}
\section{Analiza in vizualizacija podatkov}

\includegraphics{../slike/povprecna_druzina.pdf}

\section{Napredna analiza podatkov}

\includegraphics{../slike/naselja.pdf}

\end{document}
